\documentclass[a4paper]{amsart}

\usepackage{amsmath,amsthm,amssymb,verbatim,url}
\usepackage[final,pdftex,colorlinks=false,pdfborder={0 0 0}]{hyperref}

\begin{document}
\title{Documentation: hypervolume-t}

\author{Jānis Lazovskis}
\address{Riga Technical University, Riga, Latvia}
\email{janis.lazovskis\_1@rtu.lv}
\urladdr{https://www.jlazovskis.com}

\maketitle

\noindent
The software \textsc{hypervolume-t} is inspired by Benjamin Blonder's
\href{https://github.com/bblonder/hypervolume}{\textsc{hypervolume}}. Instead of using a kernel density estimator, this software uses a topologically-aware method to generate large, uniformly-distributed samples based on the input point cloud.

\section{Usage}
\noindent
After building \textsc{hypervolume-t}, you can generate point clouds as follows:

\vspace{1em}

\begin{verbatim}./hvt [options] --input filein.csv --output fileout.csv\end{verbatim}

\vspace{1em}

\vfill

\end{document}
